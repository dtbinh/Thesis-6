%!TEX root = main.tex

\chapter{Solution to degenerate symmetric linear systems}\label{app:degenetate:lin:sys}

Consider the linear system
%
\begin{equation}\label{app:eq:sys}
	\left(\begin{array}{cc}
	K & L^T \\ L & 0
	\end{array}\right)\left(\begin{array}{c}
	u \\ v \end{array}\right) = \left(\begin{array}{c}
	a \\ b
	\end{array}\right).
\end{equation}
%
We assume $K$ is a positive definite $n\times n$ matrix, $L$ is
$m\times n$. We have three different cases:
%
\begin{enumerate}
\item $\rk L = n$, which implies $n<m$
\item $\rk L = m$, which implies $m<n$
\item $\rk L < \min(n,m)$.
\end{enumerate}
%
We handle case 1:
We decompose $L$ into $L = QR$ with $Q = (Q_1, Q_2)$ and $R = (R^T,0)^T$,
with the usual orthogonal $Q$ and upper triangular $R$. In case 1 $R$ is
$r\times r$ with $r=\rk L$. We get
%
\begin{equation}\begin{split}
	L u = Q_1 R u = b\\
	\Rightarrow \underbrace{Q_1^T Q_1}_I R u = Q_1^T b\\
	\Rightarrow u = R^{-1}Q_1^T b
\end{split}\end{equation}
%
which leads to
%
\begin{equation}\begin{split}
	K R^{-1} Q_1^T b + R^T Q_1^T v = a \\
	\Rightarrow v = Q_1 R^{-T}\left(a- K R^{-1} Q_1^T b \right) + Q_2 \beta.
\end{split}\end{equation}
%
So that the solution in case one is given by
%
\begin{equation}
	\left(\begin{array}{c}
	u \\ v \end{array}\right) = \left(\begin{array}{cc}
	0 & R^{-1} Q_1^T \\ Q_1 R^{-T} & - Q_1 R^{-T} K R^{-1} Q_1^T
	\end{array}\right)\left(\begin{array}{c}
	a \\ b
	\end{array}\right) + \left(\begin{array}{c}
	0 \\ Q_2
	\end{array}\right) \beta
\end{equation}
%
In case 2 $L^T=QR$ and the equations follow in the same way.
\\[1em]
In case 3 we look at the case where $m>n$.
%
\begin{equation}\begin{split}
	K u + L^T v = a\\
	u = K^{-1} a - K^{-1}L^T v
\end{split}\end{equation}
%
into the second equation:
%
\begin{equation}\begin{split}
	L K^{-1} a - L K^{-1} L^T v = b\\
	Q_1 R K^{-1} R^T Q_1^T v = L K^{-1} a - b \\
	Q_1^T v = \left(RK^{-1}R^T\right)^{-1} Q_1^T\left(L K^{-1} a - b \right)\\
	v = Q_1 \left(RK^{-1}R^T\right)^{-1} Q_1^T\left(Q_1 R K^{-1} a - b \right) + Q_2 \beta \\
	v = Q_1 \left(RK^{-1}R^T\right)^{-1} R K^{-1} a - Q_1 \left(RK^{-1}R^T\right)^{-1} Q_1^T b + Q_2 \beta \\
\end{split}\end{equation}
%
And hence $u$ becomes
%
\begin{equation}\begin{split}
	u = K^{-1} a - K^{-1} R^T Q_1^T \left(Q_1 \left(RK^{-1}R^T\right)^{-1} Q_1^T\left(Q_1 
	R K^{-1} a - b \right) + Q_2 \beta \right) \\
	u = \left(K^{-1} - K^{-1} R^T \left(RK^{-1}R^T\right)^{-1} R K^{-1} \right) a + 
	K^{-1} R^T \left(RK^{-1}R^T\right)^{-1} Q_1^T b
\end{split}\end{equation}
%
And the closed solution to~\eqref{app:eq:sys} is given by
%
\begin{equation}\Resize{
	\left(\begin{array}{c}
	u \\ v
	\end{array}\right) = \left(\begin{array}{cc}
	K^{-\frac{1}{2}}\left(I - K^{-\frac{1}{2}} R^T \left(RK^{-1}R^T\right)^{-1} RK^{-\frac{1}{2}} \right) K^{-\frac{1}{2}} &
	K^{-1} R^T \left(RK^{-1}R^T\right)^{-1} Q_1^T \\
	Q_1 \left(RK^{-1}R^T\right)^{-1} R K^{-1} &
	-Q_1 \left(RK^{-1}R^T\right)^{-1} Q_1
	\end{array}\right)\left(\begin{array}{c}
	a \\ b
	\end{array}\right) + \left(\begin{array}{c}
	0 \\ Q_2 
	\end{array}\right) \beta}
\end{equation}