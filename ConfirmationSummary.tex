\documentclass{article}
\usepackage{color}
\usepackage[usenames]{xcolor} 
\usepackage{amssymb,mathrsfs,amsthm,amsmath,lpic,cite,natbib}
\usepackage{tikz}

\newcommand\monthheight{1.2em}
\newcommand\blockwidth{6cm}

\renewcommand\thesubsection{\arabic{subsection}}


\begin{document}

\section*{Outline}

The method of model predictive control combines optimising performance of a controlled system while respecting 
system and input constraints.
%
The aim of robust model predictive control is to provide guarantees for stability while still being able to push
a system to its physical limitations to optimise its performance.
%
Naturally, robustifying a model predictive controller leads to a more conservative controller and an obvious goal is
to try to keep the conservativeness as minimal as possible.
%
The first part of this work is dedicated in reducing this redundant conservativeness which is introduced by approximating
disturbances which are upper bounded by a fixed value.
%
In particular a robust model predictive controller using an active set solver approach is derived for linear systems
with additive disturbances and quadratic cost functions which have to satisfy linear state and input constraints.
%
The novelty is that the disturbance is considered to constrained to a set which depends on the state and the input in
a piecewise affine way.
%
The necessary set theoretic framework to handle such sets is elaborated as general as possible, as well as its numerical
treatment for which computational bounds are presented.
%
Considering disturbance bounded in a piecewise affine way allows applying linear robust model predictive control
methods for various classes of systems which are not necessarily perturbed in an additive way.
%
For example, it is shown that linear systems subject to linearly constrained multiplicative uncertainties have an
equivalent linear system with piecewise affine additive disturbances, further it is presented how a piecewise affine 
additive disturbance can be used to bound linearisation errors for nonlinear systems.
%
Hence, the usage of piecewise affine disturbance sets allows a less conservative robust treatment of nonlinear systems
than conventional constant upper bounds, while still benefiting from all properties of the active set solver approach.

Since the the active set solver approach for min-max robust control formulations is well understood, it would be beneficial
to be able to rewrite stochastic model predictive control problems, i.e. model predictive control problems where the system
is subject to probabilistic disturbances which have to satisfy probabilistic constraints, as robust model predictive control
problems.
%
A promising way to transform probabilistic constraints into deterministic ones which guarantee the satisfaction of the probabilistic
constraints is to use so called $p$-efficient sets.
%
These $p$-efficient sets have not yet been applied to model predictive control problems, but would allow robust treatment of stochastic
model predictive control problems and therefore a fast online implementation.
%
Furthermore, $p$-efficient sets are not tied to distributions with compact support, allowing treatment of a more general class of
disturbances to be numerically analysed with existing methods.
\pagebreak

\section*{Estimated thesis structure}

\setcounter{subsection}{1}

\subsection{Concepts in robust model predictive control}\label{chap:concepts}
This section aims to give an overview of concepts and ideas that are not particular to
the active-set solver approach presented later.
%
An abstract derivation of the idea to use a min-max approach to formulate robust model predictive control
problems is presented.
%
Furthermore the concepts necessary to provide stability guarantees for the robust model predictive control
scheme are explained.
%
This leads to a discussion about terminal cost terms and introduction of terminal constraints. 
%
Both of these concepts are well known and well established for nominal model predictive control, but have to 
be slightly modified for the use in a disturbed setup. 
%
The emphasis of this section is the discussion about so called robust positive invariant sets, the maximal
robust positive invariant set in particular, furthermore so called controllable sets are derived.
%
Using controllable and robust positive invariant sets as constraints and the terminal cost term in the min-max optimisation
recursion, a proof of stability is presented.

\subsection{Recursive multi-parametric quadratic programs}\label{chap:recursive:mpqp}
In this section the active-set solver for quadratic min-max programs is described.
%
This builds the foundation to the approach presented in subsequent chapters.
%
An active set solver can explicitly solve an optimisation program for which the set of active constraints is
known a priori, furthermore, using the knowledge about the structure of the solution to a quadratic program,
an active set solver returns the terms necessary to produce the numeric optimiser.
%
Having the structural components of the solution of the quadratic program allows the treatment of so called
multi-parametric quadratic programs, i.e quadratic programs which depend on a parameter.
%
It is presented how this is applied to quadratic min-max programs, such as the one arising in robust model
predictive control.
%
In order to determine the set of active constraints the structure of the optimisers once again is used to
justify a performing a line search over the initial state.
%
The procedure to perform the line search necessary is presented.

\subsection{Parametrised constraint sets}\label{chap:mrpi:sets}
One of the main contributions of this thesis is the treatment of disturbance sets that depend on a parameter,
either the state and/or the input or a scaling parameter.
%
These sets have not received sufficient treatment in the literature to use them in a min-max type robust
model predictive control problem.
%
This chapter presents some general properties of parametrised sets, convexity of the Pontryagin difference
is of particular interest.
%
For numerical tractability special care is given to sets which depend on a parameter in a piecewise affine way.
%
An example of how to use such sets to account for nonlinearities in a robust model predictive setup is presented.
%
An algorithm to compute the maximal robust positive invariant set for piecewise affine disturbance sets is presented,
together with the proof of its finite determinability.

\subsection{An active-set solver for the solution of min-max robust model predictive control problems}
In this chapter the active set solver discussed in chapter~\ref{chap:recursive:mpqp} is applied to quadratic
robust model predictive control problems.
%
The abstract method described in chapter~\ref{chap:recursive:mpqp} to find the solution of multi parametric quadratic
programs is translated for robust model predictive control problems.
%
As an illustration of the presented scheme the problem derived in the previous chapter is solved numerically and its solution
is discussed.
%
Advantages of this approach over known explicit model predictive control approaches are highlighted.

\subsection{$p$-efficient sets in stochastic model predictive control}
This chapter presents the analysis of so called $p$-efficient points.
%
Using $p$-efficient points allows guaranteeing the satisfaction of probabilistic constraints by invoking deterministic
constraints instead of probabilistic ones.
%
Properties of $p$-efficient sets are known for a particular kind of probabilistic constraints, which are not usual
in stochastic model predictive control problems with probabilistic constraints.
%
In this chapter an approximate translation for more general constraint sets is presented.
%
The aim of the analysis presented is to reformulate stochastic model predictive control programs as robust model predictive
control programs which guarantee the feasibility of the stochastic model predictive control program without being overly
conservative.

\subsection{Unbounded disturbances in stochastic model predictive control}
A standard characteristic of probabilistic variable is that, different to robust variables, their realisations are not 
necessarily bounded.
%
Justified by realism and physical meaning of variables, bounds are often assumed which also allows for numeric treatment.
%
For analytical purposes this assumption may not be necessary.
%
Hence this chapter presents analysis for stochastic model predictive control problems without compact supported disturbances,
in particular its recursive feasibility properties.
%
Utilising $p$-efficiency sets may again allow implementation of the concepts presented in this chapter.


\newpage

\section*{Estimated timetable}


\tikzset{entry/.style 2 args={
	draw,
	rectangle,
	anchor=north west,
	line width=.4pt,
	inner sep=.3333em,
	text width={\blockwidth/#2-.6666em-.4pt},
	minimum height=#1*\monthheight,
	align=center
}}

\begin{tikzpicture}[x=\blockwidth,y=-\monthheight]
\foreach \vpos/\month in 	{1/Oct 2013,2/Nov 2013,3/Dec 2013,%
							4/Jan 2014,5/Feb 2014,6/Mar 2014,%
							7/Apr 2014,8/May 2014,9/Jun 2014,%
							10/Jul 2014,11/Aug 2014,12/Sep 2014,%
							13/Oct 2014,14/Nov 2014,15/Dec 2014,%
							16/Jan 2015,17/Feb 2015,18/Mar 2015,%
							19/Apr 2015,20/May 2015,21/Jun 2015,%
							22/Jul 2015,23/Aug 2015,24/Sep 2015,%
							25/Oct 2015,26/Nov 2015,27/Dec 2015,%
							28/Jan 2016,29/Feb 2016,30/Mar 2016,%
							31/Apr 2016,32/May 2016,33/Jun 2016,%
							34/Jul 2016,35/Aug 2016,36/Sep 2016,%
							37/Oct 2016,38/Nov 2016,39/Dec 2016,%
							40/Jan 2017,41/Feb 2017,42/Mar 2017}
	\node[anchor=north east] at (1,\vpos) {\month};

	\draw (1,0) -- (1,43);
	\draw (2,0) -- (2,43);
	\draw (3,0) -- (3,43);
	\draw (1,1) -- (3,1);

	\node[anchor=north] at (1.5,0) {Research};
	\node[anchor=north] at (2.5,0) {Writing};

	\node[entry={3}{1}] at (1,1) {Familiarising with topic and predecessor's work};
	\node[entry={6}{2}] at (1,4) {Deriving theory for PWA disturbance sets};
	\node[entry={6}{2}] at (1.5,6) {Numerical implementation for inverted pendulum};

	\node[entry={3}{2}] at (2,10) {Transfer report};
	\node[entry={2}{2}] at (2.5,10) {ACC paper};

	\node[entry={2}{2}] at (1,13) {Theory for scaled constraints sets};
	\node[entry={1}{2}] at (1,17) {Computation of scaled MRPI set};
	\node[entry={5}{2}] at (1.5,15) {Implementation of Matlab interface for LRS library};

	\node[entry={4}{2}] at (2,18) {Ch. 2 - 4 of thesis};
	\node[entry={1}{2}] at (2.5,18) {CDC paper};
	\node[entry={1}{2}] at (2.5,19) {NMPC paper};

	\node[entry={1}{1}] at (2,22) {NMPC poster};

	\node[entry={4}{1}] at (1,22) {Theory for SMPC};
	\node[entry={2}{1}] at (1,26) {Reformulation of SMPC into RMPC};
	\node[entry={5}{1}] at (1,28) {Theory for disturbances with unbounded support in SMPC};

	\node[entry={2}{1}] at (2,27) {Paper about $p$-efficiency sets in SMPC};
	\node[entry={2}{2}] at (2.5,32) {Paper on results};
	\node[entry={10}{2}] at (2,33) {Writing up thesis};

\end{tikzpicture}


\end{document}